\def\comment#1{}

\parindent=0pt
\parskip=4pt
\baselineskip=11pt

\newbox\TestBox
\def\Remove #1 {\setbox\TestBox=\hbox{#1}%
        \leavevmode\rlap{\vrule height 3.5pt depth-1.75pt width\wd\TestBox}%
        \box\TestBox\ }
\def\strike{\Remove}
\def\Replaceby #1{{\bf #1}}
\def\replace{\Replaceby}



%%%  carl comments:  the initial definitions are designed to pick up just the
%%% nickname of the reference which, in his setup, is the first argument of
%%% the reference. I propose to make it the \def\rl{...} material I now use
%%% for labeled references.
%%% If the zwei-read scheme is maintained, then one would start off with the
%%% empty definition of the five \ref[A-Z], and only give their full definition
%%% just prior to the second pass.
%
% On inquiry, Peter was quite happy with a two-pass scheme, thus avoiding
% this initial reading of the references. So, we skip this initial material.

%%%%%%%%%%%%%%%%%%%%%%%%%%  block of old stuff now omitted
% \newcount\refnumb\refnumb=0
% 
% \def\ref#1#2#3#4{\global\advance\refnumb by 1
% \expandafter\xdef\csname cpl#1\endcsname{\the\refnumb}}
% 
% \def\shtref#1#2{\global\advance\refnumb by 1
% \expandafter\xdef\csname cpl#1\endcsname{\the\refnumb}}
% 
% 
% \input peter.3  %% these are the references
%%%%%%%%%%%%%%%%%%%%%%%%%%%  end of block

%%%%%%%%%%%%%%%%%%%%%%%%%%%%%%%   block of new stuff
% Here is the basic setup for the purpose of correct forward referencing,
% needed in particular if the references are read in only at the end of the
% paper. (It is taken verbatim from carl's format.tex file.)
% For this, it may be necessary to run the job twice. In any case, the
% following sets up the necessary file work:
% (1) a conditional \input, to cover the first time when the file \jobname.aux
%     does not yet exist (this may be of use elsewhere):
\newread\testfl
\def\inputifthere#1{\immediate\openin\testfl=#1
\ifeof\testfl\gdef\condfile{null}\else\gdef\condfile{#1}\fi%
\input\condfile}
% (2) make sure there is an empty  null.tex  file to read, just in case:
\newwrite\nll
\immediate\openout\nll=null.tex\closeout\nll
% (3) bring in  \jobname.aux  if it exists, then set it up for writing:
% (Recall that  \jobname  expands to the name of the file TeX thinks it is
% working on.)
\inputifthere{\jobname.aux}
\newwrite\mem
\immediate\openout\mem=\jobname.aux
%%%%%%%%%%%%%%%%%%%%%%%%%%%%%%%%  end of block

%%%%%%%%%%%%%%%%%%%%%%%%%%%%%%%% block of stuff modified
%%% \refref is used to refer to a reference; could replace by  \cite, to make 
% it more like LaTeX. Note the possibility of nonalpha stuff in a nickname.
%%% Also, much of it is used in other contexts, hence, following the rule
% that redundancy should be minimized, make those parts separate macros.


% This is the guy that places the label whose actual value is encoded in
% \griff,  making a big mark instead in case the label is undefined:
\def\plazieres{\expandafter\ifx\csname\griff\endcsname\relax%
{\vrule height15pt width15pt depth0pt}%
\else {\csname\griff\endcsname}\fi}


% This is the guy that defines and records labels whose name is contained in 
% \griff and whose value is contained in \inhalt :
\def\definieres{\expandafter\xdef\csname\griff\endcsname{\inhalt}
\def\blankkk{ }\expandafter\immediate\write\mem{%
\string\expandafter\def\string\csname%
\blankkk\griff\string\endcsname{\inhalt}}}

% This is the guy used to refer to a reference:
\def\refref#1{\edef\griff{rfl#1}[\plazieres]}
%%%%%%%%%%%%%%%%%%%%%%%%%%%%%%%% end of block of stuff modified
%%%%%%%%%%%%%%%%%%%%%%%%%%%%%%%%%%%   block modified
%% Here he sets up automatic sequencing of proclaim items and equations.
%% He uses the same mechanism for generating nicknames for them, but doesn't
%% seem to have a means for forward referencing. Will propose my mechanism
%% for him, as that will also obviate the two passes through the references.

\newcount\proclaimnumber
\def\cp#1{\global\advance\proclaimnumber by 1
\edef\griff{cpl#1}\edef\inhalt{\the\proclaimnumber}\definieres
{\bf\inhalt}}
\def\cpref#1{\edef\griff{cpl#1}\plazieres}

\newcount\equationnumber
\def\eq#1{\global\advance\equationnumber by 1
\edef\griff{eql#1}\edef\inhalt{\the\equationnumber}\definieres
\eqno(\inhalt)}
\def\eqref#1{\edef\griff{eql#1}(\plazieres)}
%%%%%%%%%%%%%%%%%%%%%%%%%%%%%%%%%%%  end of block modified



\def\today{\ifcase\month\or January\or February\or March\or April\or
May\or June\or July\or August\or September\or October\or November\or
December\fi \space\number\day, \number\year}

\def\query#1{\global\advance\fnumb by
1{\footnote{${}^{-\the\fnumb-}$}{{\rm #1}}}} \newcount\fnumb

\newcount\figurenumb\figurenumb=0
\def\fp#1{\global\advance\figurenumb by 1
\expandafter\xdef\csname fpl#1\endcsname{\the\figurenumb}
{\bf\the\figurenumb}}
\def\figref#1{\expandafter\ifx\csname fpl#1\endcsname\relax
 {{\vrule height15pt width15pt depth0pt}}
 \else
   {\csname fpl#1\endcsname}\fi}

\comment{Parameters of \figure:
 #1 caption
 #2 horizontal size (in inches)
 #3 vertical size (in inches)
 #4 file name
 #5 identifying phrase}


\def\figure#1#2#3#4#5{
\midinsert
\smallskip
\FIGPLOT{#4}{{\bf Figure\fp{#5}.} #1}{#2}{#3}
\smallskip
\endinsert
}

\newdimen\figoffset
\def\FIGPLOT#1#2#3#4{%
  % Arg 1 = EPS file to plot
  % Arg 2 = figure caption
  % Arg 3 = width in inches
  % Arg 4 = height in inches
      \vskip #4in%
      \vskip\baselineskip%
      \figoffset=\hsize%
      \advance\figoffset by -#3in%
      \divide\figoffset by 2%
      \advance\figoffset by -\parindent
      \hskip\figoffset%
      \NormalGraph{#3}{#4}{#1}%
      \hskip-\figoffset%
      \vskip\baselineskip%
      \tolerance=6000
%     \emergencystretch=3pt%
      \centerline{#2.}
}
\def\NormalGraph#1#2#3{%
      \special{%
  language "PostScript",
  position "bottom left",
  literal "/SX {#1 72 mul BoxWidth div} def
    /SY {#2 72 mul BoxHeight div} def
    1 SX sub BoxLLX mul
    1 SY sub BoxLLY mul
    translate
    SX SY scale
    \the\mag()pop 1000 div dup scale",
  include "#3"
}}







\def\boxit#1{\lower8pt\vbox{\hrule height2pt \hbox{\vrule width2pt
\kern8pt \vbox{\kern8pt\hbox{$\displaystyle
#1$}\kern8pt}\kern8pt\vrule width2pt}\hrule height2pt }}

\def\Bezier{B\'ezier}

\def\d{\hbox{d}}

\def\smatrix#1{\left[\matrix{#1}\right]}
\def\R{\hbox{I\kern-.2em\hbox{R}}}
\def\C{\hbox{I\kern-.6em\hbox{C}}}
\def\sect#1{\bigskip{\lg #1}\nobreak\medskip}
\def\set#1{\left\{#1\right\}}
\def\ip#1#2{\left<#1,#2\right>}

\def\rdots{ \mathinner {\mkern 1mu\raise 1pt \vbox {\kern 7pt \hbox
{.}}\mkern 2mu \raise 4pt \hbox {.}\mkern 2mu\raise 7pt \hbox
{.}\mkern 1mu}}


\font\lg=cmbx10 at 12pt
\font\huge=cmbx10 at 18pt
\font\sm=cmtt10 at 8pt


% -*-TeX-*-
% <TEX.INPUTS>LISTING.TEX.1,  4-Nov-86 16:19:19, Edit by BEEBE
% Implement the listing macro from the TeXbook (p. 381)
% Usage:
%	\input listing	% define these macros
%	\listing{filename}	 % produce a line-numbered verbatim listing
%
\newcount\lineno
\def\uncatcodespecials{\def\do##1{\catcode`##1=12 }\dospecials}
\def\setupverbatim{\tt
  \lineno=0%
  \def\par{\leavevmode\endgraf}%

  %? The following does not work--is there a TeXbook error?
  %? \catcode`\`=\active{\catcode`\`=\active\gdef`{\relax\lq}}%

  \overfullrule0pt	% do not mark long lines with overfull box rule
  \obeylines
  \uncatcodespecials
  \obeyspaces
  \everypar{\advance\lineno by 1 \llap{\sevenrm\the\lineno\ \ }}%
}%
{\obeyspaces\global\let =\ }% let active space=control space
\def\listing#1{\begingroup\setupverbatim\baselineskip=9pt\parskip=0pt\input#1 \endgroup\smallskip}%



\def\beginemail{\bigskip
\begingroup\tt \frenchspacing \raggedright \obeylines \obeyspaces
\baselineskip=10pt \parskip=0pt}

\def\endemail{\endgroup}


\def\h{\hfill}

\def\remark#1{#1\par}
\def\electrosubmitted{This is a review text file submitted electronically 
                      to MR.\par}
\def\beginrev{}
\def\reviewer{\par{\bf Reviewer: }}
\def\reviewernum{\par{\bf Reviewer number: }}
\def\address{\par{\bf Address: }\par
  \begingroup\leftskip=16pt\parskip=0pt\obeylines}
\def\author{\endgroup\par{\bf Author: }}
\def\shorttitle{\par{\bf Short title: }}
\def\mcno{\par{\bf Control number: }}
\def\mmrno{}
\def\rpclass{\par{\bf Primary classification: }}
\def\rsclass{\par{\bf Secondary classification(s): }}
\def\revtext{\par{\bf Review text: }\smallskip}
\def\remarks#1{}                                     
\def\endrev{}


\def\pt{\smallskip\item{$\bullet$}}

\def\ppt{\itemitem{$-$}}

 \newcount\prbnumb % ........count reference numbers \prbnumb = 0

 \def\prb#1#2{{{\global\advance\prbnumb by 1}\medskip \item{{\lg
-\the\prbnumb-}}{\bf (#1.)} #2}}

\def\rank{\hbox{rank}}

\def\span{\hbox{span}}

\def\disc{\smallskip \noindent {\lg Discussion:}\nobreak\smallskip\nobreak}

\def\where{\quad\hbox{where}\quad}

\def\and{\quad\hbox{and}\quad}

%\def\or{\quad\hbox{or}\quad}

\def\O{{\cal O}}

\def\implies{\qquad\Longrightarrow\qquad}

\newcount\figurenumb\figurenumb=0
\def\fp#1{\global\advance\figurenumb by 1
\expandafter\xdef\csname fpl#1\endcsname{\the\figurenumb}
{\bf\the\figurenumb}}
\def\figref#1{\expandafter\ifx\csname fpl#1\endcsname\relax
 {{\vrule height15pt width15pt depth0pt}}
 \else
   {\csname fpl#1\endcsname}\fi}

\comment{Parameters of \figure:
 #1 caption
 #2 horizontal size (in inches)
 #3 vertical size (in inches)
 #4 file name
 #5 identifying phrase}


\def\figure#1#2#3#4#5{
\midinsert
\bigskip
\FIGPLOT{#4}{{\bf Figure\fp{#5}.} #1}{#2}{#3}
\endinsert}

\def\Doublefigure#1#2#3#4#5#6{
\midinsert
\smallskip
\doubleFIGPLOT{#4}{#5}{{\bf Figure\fp{#6}.} #1}{#2}{#3}
\smallskip
\endinsert}

\def\figure#1#2#3#4#5{
\midinsert
\bigskip
\FIGPLOT{#4}{{\bf Figure\fp{#5}.} #1}{#2}{#3}
\endinsert}

\newdimen\figoffset

\input epsf.sty


\def\FIGPLOT#1#2#3#4{%
  % Arg 1 = EPS file to plot
  % Arg 2 = figure caption
  % Arg 3 = width in inches
  % Arg 4 = height in inches
%%:      \vskip #4in%
%%:      \vskip\baselineskip%
%%:      \figoffset=\hsize%
%%:      \advance\figoffset by -#3in%
%%:      \divide\figoffset by 2%
%%:      \advance\figoffset by -\parindent
%%:      \hskip\figoffset%
%%:      \NormalGraph{#3}{#4}{#1}%
%%:      \hskip-\figoffset%

%%: [28-Sep-1999] New code that works with both dvialw and dvips, and 
%%:               always preserves aspect ratio: the height argument
%%:		  is ignored to achieve this, and the plot is always
%%:		  centered on the line; it is is wider than the line,
%%:		  it will extend into the margins equally.
    \allowbreak			%% page break before figure is okay
%%    \bigskip			%% this much space before (and later, after) figure
    \epsfframefalse		%% change to \epsfframefalse to turn off frames
    \epsfclipon			%% clipping ensures that the figure cannot extend 
				%% outside its BoundingBox		
    \noindent			%% suppress paragraph indentation
    \centerline{\hss\hbox{\epsfxsize=#3in\epsffile{#1}}\hss}% plot the centered figure
    \nobreak			%% forbid page break 
    \nobreak
%%:   \vskip\baselineskip%
%%:      \tolerance=6000
%%:     \emergencystretch=3pt%
    \smallskip			%% this much space between figure and caption
    \centerline{#2.}		%% centered figure caption
%%    \bigskip			%% this much space after (and earlier, before) figure
    \allowbreak			%% page break after figure is okay
}
%%: \def\NormalGraph#1#2#3{%
%%:       \special{%
%%:   language "PostScript",
%%:   position "bottom left",
%%:   literal "/SX {#1 72 mul BoxWidth div} def
%%:     /SY {#2 72 mul BoxHeight div} def
%%:     1 SX sub BoxLLX mul
%%:     1 SY sub BoxLLY mul
%%:     translate
%%:     SX SY scale
%%:     \the\mag()pop 1000 div dup scale",
%%:   include "#3"
%%: }}

\def\doubleFIGPLOT#1#2#3#4#5{%
  % Arg 1 = EPS file to plot
  % Arg 2 = 2nd EPS file to plot
  % Arg 3 = figure caption
  % Arg 4 = width in inches
  % Arg 5 = height in inches
%%:      \vskip #5in%
%%:      \vskip\baselineskip%
%%:      \figoffset=\hsize%
%%:      \advance\figoffset by -#3in%
%%:      \divide\figoffset by 2%
%%:      \advance\figoffset by -\parindent
%%:      \hskip\figoffset%
%%:      \NormalGraph{#3}{#5}{#1}%
%%:      \hskip-\figoffset%

%%: [28-Sep-1999] New code that works with both dvialw and dvips, and 
%%:               always preserves aspect ratio: the height argument
%%:		  is ignored to achieve this, and the plot is always
%%:		  centered on the line; it is is wider than the line,
%%:		  it will extend into the margins equally.
    \allowbreak			%% page break before figure is okay
%%    \bigskip			%% this much space before (and later, after) figure
    \epsfframefalse		%% change to \epsfframefalse to turn off frames
    \epsfclipon			%% clipping ensures that the figure cannot extend 
				%% outside its BoundingBox		
    \noindent			%% suppress paragraph indentation
    \centerline{\hss\hbox{\epsfxsize=#4in\epsffile{#1}~\hfill\epsfxsize=#4in\epsffile{#2}}\hss}% plot the centered figure
    \nobreak			%% forbid page break 
    \nobreak
%%:   \vskip\baselineskip%
%%:      \tolerance=6000
%%:     \emergencystretch=3pt%
    \smallskip			%% this much space between figure and caption
    \centerline{#3.}		%% centered figure caption
%%    \bigskip			%% this much space after (and earlier, before) figure
    \allowbreak			%% page break after figure is okay
}
%%: \def\NormalGraph#1#2#3{%
%%:       \special{%
%%:   language "PostScript",
%%:   position "bottom left",
%%:   literal "/SX {#1 72 mul BoxWidth div} def
%%:     /SY {#2 72 mul BoxHeight div} def
%%:     1 SX sub BoxLLX mul
%%:     1 SY sub BoxLLY mul
%%:     translate
%%:     SX SY scale
%%:     \the\mag()pop 1000 div dup scale",
%%:   include "#3"
%%: }}

\def\ww{{\bf WeBWorK}}

    \def\LaTeX{L\kern-.36em\raise.3ex\hbox{\sc \uppercasesc a}\kern-.15em\TeX}%

    \def\LaTeX{L\kern-.36em\raise.3ex\hbox{\rm a}\kern-.05em\TeX}%
